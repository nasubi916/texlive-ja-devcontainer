\documentclass[uplatex,dvipdfmx]{jlreq}
\begin{document}

日本で漱石が「我が輩は猫である」を発表したころ、
ドイツではAlbert Einsteinが特殊相対論を発表した。

この理論によれば、エネルギー $E$ と質量 $m$ は
\begin{equation}
  E = mc^{2}
\end{equation}
で関係づけられる。ここに $c$ は光速度で、
\begin{equation}
  c = 299{,}792{,}458 \, \mathrm{m/s}
\end{equation}
である。

次のような数式も綺麗に出力出来ます。
\[
  \frac{\pi}{2} =
  \left( \int_{0}^{\infty} \frac{\sin x}{\sqrt{x}} \, dx \right)^{2} =
  \sum_{k=0}^{\infty} \frac{(2k)!}{2^{2k}(k!)^2} \frac{1}{2k+1} =
  \prod_{k=1}^{\infty} \frac{4k^2}{4k^2 - 1}
\]

\end{document}
